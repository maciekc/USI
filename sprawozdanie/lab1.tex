\chapter{Wstęp}

\section{Cel zajęć}

Celem projektu wykonywanego w ramach zajęć z przedmiotu \textit{Układy Sterowania Inteligentnego} było zaprojektowanie regulatora dla manipulatora, którego zadaniem było zabranie szklanki z wodą z jednego miejsca i odstawienie jej w innym miejscu. Przemieszczał się on w płaszczyźnie poziomej. Napędzany był jednym silnikiem prądu stałego.

Projektowany regulator miał być układem sterowania inteligentnego. W ramach tego projektu należało również przeprowadzić porównanie inteligentnych algorytmów sterowania (np. sieć neuronowa lub regulator fuzzy) z regulatorami klasycznymi (np. PID, LQ lub czasooptymalny).

\section{Obiekt sterowania}

Obiektem sterowania był manipulator, który przenosił szklankę z wodą, a następnie powracał do położenia początkowego. Musiał on przenieść szklankę bez wylewania jej zawartości. Takie zadanie nie jest tożsame z samym pozycjonowaniem manipulatora. Wymaganie, by nie wylać wody narzuca ograniczenia na ruch obiektu. Musi poruszać się on z odpowiednio małym przyspieszeniem, gdy trzyma szklankę z wodą. Ograniczenie to nie jest jednak ważne, gdy powraca do położenia początkowego. Jest więc oczywiste, że dla ruchu w obie strony powinny zostać użyte inne regulatory. Pierwszy z regulatorów powinien zapewnić spełnienie następującego ograniczenia:
\begin{equation}
\ddot \phi(t) \leqslant \epsilon_{max}
\end{equation}
\noindent gdzie:\newline
\(\phi\) jest położeniem kątowym manipulatora,\newline
\(\epsilon_{max}\) jest maksymalnym przyspieszeniem kątowym.

\paragraph*{}
Oprócz tego konieczne było zdefiniowanie funkcji celu, dla którą minimalizować miał projektowany regulator. Zdecydowano, że będzie ona miała następującą postać:
\begin{equation}
J(\phi,r,t)=\int\limits_{0}^{\infty} (\phi(t)-r(t))^2dt+\int\limits_{0}^{\infty} u(t)^2dt
\end{equation}
\noindent gdzie:\newline
\(\phi\) jest położeniem kątowym manipulatora,\newline
\(r\) jest zadaną pozycją manipulatora,\newline
\(u\) jest sterowaniem podawanym na obiekt.

\paragraph*{}
Należy zwrócić uwagę, że postanowiono zaniedbać opory ruchu. W związku z tym jedynym momentem siły działającym na manipulator był moment pochodzący od silnika prądu stałego. W dalszej części projektu możliwe jest zmodyfikowanie zadania w taki sposób, by wziąć pod uwagę opory związane z ruchem obrotowym manipulatora.