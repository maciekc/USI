\chapter{Regulator}
\label{cha:regulator}

\section{Zaproponowany regulator}
Do pozycjonowania manipulatora zaproponowany został regulator składający się z dwóch równolegle połączonych regulatorów PID. Kiedy trzymana jest pełna szklanka, to na wyjście przekazywane jest sterowanie z pierwszego regulatora, a kiedy jest pusta to z drugiego. Regulatorom postanowiono zadać inne nastawy, takie, by ograniczyć przyspieszenie kątowe w sytuacji, gdy trzymana jest pełna szklanka.Ma to na celu spełnienie warunku, by przyspieszenie było małe, aby nie wylać wody.

Pozycja zadana podawana na regulator manipulatora miała postać funkcji prostokątnej. Stwierdzono jednak, że z uwagi na ograniczenie przyspieszenia, czas pozycji zadanej dla ruchu z pełną szklanką powinien być dłuższy.