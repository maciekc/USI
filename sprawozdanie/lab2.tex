\chapter{Model matematyczny}

\section{Model silnika elektrycznego prądu stałego}
Silnik elektryczny odpowiedzialny za poruszanie ramieniem został zamodelowany jako obiekt inercyjny pierwszego rzędu. Równania \ref{ele1} i \ref{ele2} opisują zależność generowanego momentu obrotowego od prądu.
\begin{equation}\label{ele1}
M(t) = k_e \cdot i(t)
\end{equation}
\begin{equation}\label{ele2}
U(t) = i(t) \cdot R + L \cdot \frac{di(t)}{dt}
\end{equation}
gdzie:\\
$M(t)$ - moment generowany przez silnik,\\
$i(t)$ - prąd elektryczny,\\
$k_e$ - stała elektryczna silnika,\\
$L$ - indukcyjność silnika.\\
%
\section{Model matematyczny obiektu}
Równania mechaniczne opisujące dynamikę całego układu mają postać:\\
\begin{equation}\label{r1}
\frac{d^2 \alpha(t)}{dt^2} \cdot J = k_e \cdot i(t) 
\end{equation}
\begin{equation}\label{r2}
U(t) = i(t) \cdot R + \frac{d i(t)}{dt} \cdot L
\end{equation}
gdzie:\\
$\alpha$ - kąt wychylenia,\\
$J$ - moment bezwładności ramienia,\\
$U(t)$ - napięcie podawane na silnik,\\

W przyjętym modelu obiektu założono że wielkością sterującą jest napięcie podawane na silnik, a wyjściową kąt wychylenia ramienia.\\
Na podstawie równań \ref{r1} i \ref{r2} zapisano model matematyczny w postać równań stanu przyjmując następujące zmienne stanu:\\
$x_1$ - prąd silnika\\
$x_2$ -  położenie kątowe ramienia\\
$x_3$ - prędkość kątowa ramienia\\
\begin{equation}\label{key}
\dot {x_1} = -x_1  \cdot \frac{R}{L}
\end{equation}
\begin{equation}\label{key}
\dot {x_2} = x_3
\end{equation}
\begin{equation}\label{key}
\dot {x_3} = \frac{k_e}{J} \cdot J
\end{equation}