\chapter{Wnioski}
Przeprowadzenie badań opisanych w niniejszym sprawozdaniu pozwoliło zapoznać się, z innymi niż dotychczas poznane, rodzajami regulatorów. Zaprojektowanie od podstaw regulatora neuronowego jak i również rozmytego pozwoliło lepiej zrozumieć zasadę ich działania. Dzięki wybraniu, na potrzeby symulacji, prostego obiektu drugiego rzędu mogliśmy w łatwy sposób badać wpływ wprowadzanych w strukturze regulatora modyfikacji na działanie całego systemu. Zamieszczone w sprawozdaniu porównania klasycznych regulatorów z regulatorem neuronowym oraz rozmytym pokazało, że każde z rozważanych podejść daje porównywalne wyniki i może być zastosowane w praktyce. \\
Podsumowując, regulator neuronowy jak i rozmyty spełnił swoją rolę, jednak aby uzyskać zbliżone wyniki do regulatorów klasycznych należało poświęcić więcej czasu na dobraniu odpowiedniej struktury i parametrów regulatora. Wykorzystanie sieci neuronowych bąd\'z też logiki rozmytej może mieć swoje uzasadnienie w przypadku bardziej złożonych systemów. Dla prostych obiektów pierwszego i drugiego rzędu może być ciężko uzyskać lepszą jakość regulacji niż w przypadku regulator PID lub LQ bez wykorzystania dedykowanych narzędzi \textit{Matlab-a}.