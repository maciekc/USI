\documentclass[12pt]{report}
\usepackage[utf8]{inputenc}
\usepackage{amsmath}
\usepackage{amsfonts}
\usepackage{amssymb}
\usepackage[pdftex]{graphicx}
\usepackage{polski}

\usepackage{placeins}
\usepackage{epstopdf}
\usepackage{mathtools}
\usepackage{amsthm}
\usepackage{float}
\usepackage{tabularx}
\usepackage{titlesec}

\titleformat{\chapter}
{\Large\bfseries} % format
{}                % label
{0pt}             % sep
{\huge}           % before-code


\usepackage[left=2.50cm, right=2.50cm, top=2.50cm, bottom=2.50cm]{geometry}

\newcolumntype{C}[1]{>{\hsize=#1\hsize\centering\arraybackslash}X}
\newcolumntype{M}[1]{>{\centering\arraybackslash}m{#1}}

%opening
\title{\textbf{Laboratorium problemowe \\ Układy Sterowania Inteligentnego.}}
\author{Maciej Cebula \\Marcin Kowalczyk \\ Daniel Rubak}
\date{}
\begin{document}
	
%	\fancypagestyle{plain}
%	{
%		% Usuń nagłówek i stopkę
%		\fancyhf{}
%		% Usuń linie.
%		\renewcommand{\headrulewidth}{0pt}
%		\renewcommand{\footrulewidth}{0pt}
%	}

	
	\setcounter{tocdepth}{2}
	
	\maketitle
	\tableofcontents
	\clearpage
		
		\renewcommand{\tablename}{Tabela}
		\renewcommand{\figurename}{Rys.}
		
	\chapter{Wstęp}
\label{cha:wstep}

\section{Cel zajęć}

Celem projektu wykonywanego w ramach zajęć z przedmiotu \textit{Układy Sterowania Inteligentnego} było zaprojektowanie regulatora dla manipulatora, którego zadaniem było zabranie szklanki z wodą z jednego miejsca i odstawienie jej w innym miejscu. Przemieszczał się on w płaszczyźnie poziomej. Napędzany był jednym silnikiem prądu stałego.

Projektowany regulator miał być układem sterowania inteligentnego. W ramach tego projektu należało również przeprowadzić porównanie inteligentnych algorytmów sterowania (np. sieć neuronowa lub regulator fuzzy) z regulatorami klasycznymi (np. PID, LQ lub czasooptymalny).

\section{Obiekt sterowania}

Obiektem sterowania był manipulator, który przenosił szklankę z wodą, a następnie powracał do położenia początkowego. Musiał on przenieść szklankę bez wylewania jej zawartości. Takie zadanie nie jest tożsame z samym pozycjonowaniem manipulatora. Wymaganie, by nie wylać wody, narzuca ograniczenia na ruch obiektu. Musi poruszać się on z odpowiednio małym przyspieszeniem, gdy trzyma szklankę z wodą. Ograniczenie to nie jest jednak ważne, gdy powraca do położenia początkowego. Jest więc oczywiste, że dla ruchu w obie strony powinny zostać użyte inne regulatory. Pierwszy z regulatorów powinien zapewnić spełnienie następującego ograniczenia:
\begin{equation}
\ddot \phi(t) \leqslant \epsilon_{max}
\end{equation}
\noindent gdzie:\newline
\(\phi\) jest położeniem kątowym manipulatora,\newline
\(\epsilon_{max}\) jest maksymalnym przyspieszeniem kątowym.

%\paragraph*{}
\section{Wska\'zniki jakość}
Aby móc porównać ze sobą różne struktury regulatorów  konieczne było zdefiniowanie wska\'zników jakości, które miały minimalizować działanie regulatora według przyjętych kryteriów. Zdecydowano, że będą one następującej postaci:
\begin{enumerate}
	\item całka z kwadratu uchybu regulacji $J_1 = \int_{0}^{tk}e(t)^2 dt \  [rad^2 \cdot s]$
	\item  wska\'znik energetyczny $J_2 = \int_{0}^{tk}u(t)^2 dt \ [V^2 \cdot s]$
	\item  suma powyższych wska\'zników $J_3 = J_1 + J _2$
\end{enumerate}
\begin{equation}
%J(\phi,r,t)=\int\limits_{0}^{\infty} (\phi(t)-r(t))^2dt+\int\limits_{0}^{\infty} u(t)^2dt
\end{equation}
\noindent gdzie:\newline
$e(t) = r - \phi(t)$ to uchyb regulacji\\
\(\phi\) jest położeniem kątowym manipulatora,\newline
\(r\) jest zadaną pozycją manipulatora,\newline
\(u\) jest sterowaniem podawanym na obiekt.

\paragraph*{}
Należy zwrócić uwagę, że postanowiono zaniedbać opory ruchu. W związku z tym jedynym momentem siły działającym na manipulator był moment pochodzący od silnika prądu stałego. W dalszej części projektu możliwe jest zmodyfikowanie zadania w taki sposób, by wziąć pod uwagę opory związane z ruchem obrotowym manipulatora.
	
	\begin{thebibliography}{}
	
	%przykład 
	
	\bibitem{LP}Cebula M., Kowalczyk M., Rubak D.: 
	\emph{Model helikoptera. Laboratorium Problemowe 2}, Kraków 2018
	

	
\end{thebibliography}
	
\end{document}